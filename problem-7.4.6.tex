\documentclass[12pt]{article}
\usepackage[top=1.0in, bottom=1.0in, left=1.0in, right=2.5in]{geometry}
\usepackage{fancyhdr, textpos, faktor, enumerate}
\usepackage{amsmath, amssymb, amsthm}
\usepackage{ebgaramond-maths}

\pagestyle{fancy}
\fancyhf{}
\setlength\headheight{28pt}
\setlength\headwidth{7in}
\setlength{\TPHorizModule}{1pt}
\setlength{\TPVertModule}{1pt}
\lhead{\textsc{Number Theory}}
\rhead{\textsc{March 10, 2020}}
\chead{\textsc{Rob Ireton}}

\newenvironment{exercise}[2]{\begin{textblock}{32}[1,0](0,#2)\noindent#1\end{textblock}}{\vspace{1in}}

\begin{document}

\begin{exercise}{7.4.6}{6}
  {\noindent}Verify the formula $\sum_{d=1}^{n}{\phi(d)\lfloor n/d \rfloor} = n(n+1)/2$ for any positive integer n.

  \begin{proof}
    Let $F(m) = \sum\limits_{d \mid m} \phi(d)$. Then
    \begin{align*}
        \sum_{d=1}^n \phi(d) \left\lfloor \frac{n}{d} \right\rfloor &= \sum_{k=1}^n F(k)  & \textit{ Theorem 6.11}\\
        &= \sum_{k=1}^n \sum_{r \mid k} \phi(r) & \\
        &= \sum_{k=1}^n k  & \textit{ Theorem 7.6}\\
        &= 1 + 2 + \cdots + n & \\
        &= \frac{n(n+1)}{2} & 
    \end{align*}
    which is the desired result.
  \end{proof}
\end{exercise}

\end{document}
