\documentclass[12pt]{article}
\usepackage[top=1.0in, bottom=1.0in, left=1.0in, right=2.5in]{geometry}
\usepackage{fancyhdr, textpos, faktor, enumerate}
\usepackage{amsmath, amssymb, amsthm}
\usepackage{multicol}
\usepackage{verbatimbox}
\usepackage{ebgaramond-maths}

\pagestyle{fancy}
\fancyhf{}
\setlength\headheight{16pt}
\setlength\headwidth{5in}
\setlength{\TPHorizModule}{1pt}
\setlength{\TPVertModule}{1pt}
\setlength{\tabcolsep}{1em}
\linespread{1.2}
\lhead{\textsc{Number Theory}}
\rhead{\textsc{Midterm}}
\chead{\textsc{Rob Ireton}}

\begin{document}

\noindent What if, instead of writing numbers in decimal or binary form with each place representing successive powers of the base, we wrote numbers with each place representing the powers of the prime numbers? Instead of the units, tens, hundreds, etc. places, we have the twos, threes, fives, etc. places. The Fundamental Theorem of Arithmetic guarantees that we can write every positive integer greater than one in this way. For example, $20 = 5^1 \cdot 3^0 \cdot 2^2$ so we would write $(20)_{10}$ as $(102)_p$. Here are what the first 36 positive integers would look like:

\ttfamily \small
\addvbuffer[12pt]{\begin{tabular}{ r r | r r | r r }
 1 &           0 & 13 &      100000 & 25 &         200 \\
 2 &           1 & 14 &        1001 & 26 &      100001 \\
 3 &          10 & 15 &         110 & 27 &          30 \\
 4 &           2 & 16 &           4 & 28 &        1002 \\
 5 &         100 & 17 &     1000000 & 29 &  1000000000 \\
 6 &          11 & 18 &          21 & 30 &         111 \\
 7 &        1000 & 19 &    10000000 & 31 & 10000000000 \\
 8 &           3 & 20 &         102 & 32 &           5 \\
 9 &          20 & 21 &        1010 & 33 &       10010 \\
10 &         101 & 22 &       10001 & 34 &     1000001 \\
11 &       10000 & 23 &   100000000 & 35 &        1100 \\
12 &          12 & 24 &          13 & 36 &          22 \\
\end{tabular}}
\normalfont \normalsize

\section*{Some Observations}
\noindent The FTA only applies to positive integers \textit{greater} than one, but we can represent one as $2^0$ or $(0)_p$. To represent $0$ we would need to introduce an entirely separate character. $(12)_{10}$ is also $(12)_p$, and is the only instance of this for numbers less than $1024$. We need digits beyond $0$\textendash$9$ once we get to $1024 = 2^{10}$. We can distinguish even from odd numbers by inspecting the right-most digit. Is it a $0$? Then the number is odd; otherwise, it's even. Square numbers have only $2$s and $0$s; cubes have only $3$s and $0$s, etc. Square-free numbers have only $1$s and $0$s.

Primes are easy to spot. They all look like $1$ followed by zero or more $0$s. They do get really long (number of digits) really quickly, however. We can get a sense by looking at the function $\pi(n)$, the number of primes not exceeding $n$. $\pi(36) = 11$, so we need $11$ places to show the list of $36$ numbers above. (Prime $(31)_{10}$ is a $1$ followed by ten $0$s.) $\pi(100) = 25$ and $\pi(1000) = 168$. The Prime Number Theorem tells us that we may need as many as (approximately) $n \log n$ digits to display a number $n$ in this notation. The notation doesn't even help us find primes. Sure, you can take a prime and get the next one by adding a $0$ onto the end, but you have to know what prime that new place on the left represents before you can interpret the digits.

Multiplication is easy: just add digits. $(7)_{10} \times (18)_{10} = (126)_{10}$ becomes $(1000)_p \times (21)_p = (1021)_p$. $(9)_{10} \times (24)_{10} = (216)_{10}$ becomes $(20)_p \times (13)_p = (33)_p$. However, there is no \textit{carry-over} from place to place. To multiply $(6)_p$ by $(15)_p$ [$64 \times 96 = 6144$ in decimal] will require an additional symbol for the $2$s place: something like $(1\mathrm{B})_p$. In contrast to notations like decimal and binary, which use a fixed repertoire of digits, this notation will need infinitely-many symbols. The Unicode Consortium will not be pleased\ldots.

Addition, however, is a problem. Besides not having an identity element (no zero), the well-known algorithms just don't apply. $(1)_p + (10)_p = (100)_p$ [$2+3=5$] may look promising, but things like $(21)_p + (102)_p = (10000001)_p$ [$18+20=38$] and $(1000000)_p + (10000000)_p = (22)_p$ [$17+19=36$] induce despair. We have also given up being able to sort numbers lexicographically. Numbers with fewer digits are not necessarily smaller than numbers with more. Numbers with the same number of digits can't be reliably sorted without doing calculations. For example, $(1000)_p < (20)_p < (12)_p < (5)_p$ is true.

\section*{Divisibility}
\noindent If we multiply by adding digit-by-digit, then to divide, we subtract digit-by-digit. The digits in our notation are all non-negative integers, so to divide $b$ by $a$, all of the digits of $a$ must be less than or equal to all of the corresponding digits of $b$, ensuring that all of the differences are non-negative. More formally, $a \mid b$ if and only if every digit of $a$ is less than or equal to the corresponding digit of $b$. (We may, of course, pad our numerals on the left with $0$s as needed.) $(11)_p \mid (13)_p$ ($6 \mid 24$) because $1 = 1$ and $1 < 3$. Doing the division shows that $(11)_p \div (13)_p = (2)_p$ ($4$ in decimal). $(20)_p \nmid (12)_p$ ($9 \nmid 12$) because $2 > 1$.

A theorem like, “If $p$ is prime and $p \mid ab$, then $p \mid a$ or $p \mid b$.” has a convenient explanation. “$p$ is prime” means that it looks like a $1$ followed by only $0$s. “$p \mid ab$” means that the digit in $ab$ corresponding to the $1$ in $p$ must be at least $1$. That digit is itself the sum of the corresponding digits of $a$ and $b$. For that digit to be at least one, either $a$ or $b$ has that digit be at least one. That is, $p \mid a$ or $p \mid b$.

\end{document}
