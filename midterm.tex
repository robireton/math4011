\documentclass[12pt]{article}
\usepackage[top=1.0in, bottom=1.0in, left=1.0in, right=2.5in]{geometry}
\usepackage{fancyhdr, textpos, faktor, enumerate}
\usepackage{amsmath, amssymb, amsthm}
\usepackage{multicol}
\usepackage{verbatimbox}
\usepackage{listings}
\usepackage{ebgaramond-maths}

\pagestyle{fancy}
\fancyhf{}
\setlength\headheight{16pt}
\setlength\headwidth{5in}
\setlength{\TPHorizModule}{1pt}
\setlength{\TPVertModule}{1pt}
\setlength{\tabcolsep}{1em}
\linespread{1.2}
\lhead{\textsc{Number Theory}}
\rhead{\textsc{Midterm}}
\chead{\textsc{Rob Ireton}}
\rfoot{\thepage}

\newcommand{\ppn}[1]{(#1)_p}

\begin{document}

\noindent What if, instead of writing numbers in decimal or binary form with each place representing successive powers of the base, we wrote numbers with each place representing the powers of the prime numbers? Instead of the units, tens, hundreds, etc. places, we have the twos, threes, fives, etc. places. The Fundamental Theorem of Arithmetic guarantees that we can write every positive integer greater than one in this way. For example, $20 = 5^1 \cdot 3^0 \cdot 2^2$ so we would write $(20)_{10}$ as $\ppn{102}$. Here are what the first 36 positive integers would look like:

\ttfamily \small
\addvbuffer[12pt]{\begin{tabular}{ r r | r r | r r }
 1 &           0 & 13 &      100000 & 25 &         200 \\
 2 &           1 & 14 &        1001 & 26 &      100001 \\
 3 &          10 & 15 &         110 & 27 &          30 \\
 4 &           2 & 16 &           4 & 28 &        1002 \\
 5 &         100 & 17 &     1000000 & 29 &  1000000000 \\
 6 &          11 & 18 &          21 & 30 &         111 \\
 7 &        1000 & 19 &    10000000 & 31 & 10000000000 \\
 8 &           3 & 20 &         102 & 32 &           5 \\
 9 &          20 & 21 &        1010 & 33 &       10010 \\
10 &         101 & 22 &       10001 & 34 &     1000001 \\
11 &       10000 & 23 &   100000000 & 35 &        1100 \\
12 &          12 & 24 &          13 & 36 &          22 \\
\end{tabular}}
\normalfont \normalsize

\section*{Some Observations}
The FTA only applies to positive integers \textit{greater} than one, but we can represent one itself as $2^0$ or $\ppn{0}$. To represent zero we would need to introduce an entirely separate character. $(12)_{10}$ is also $\ppn{12}$, and is the only instance of this for numbers less than $1024$. We need digits beyond $0$\textendash$9$ once we get to $1024 = 2^{10}$. We can distinguish even from odd numbers by inspecting the right-most digit. Is it a $0$? Then the number is odd; otherwise, it's even. Square numbers have only $2$s and $0$s; cubes have only $3$s and $0$s, etc. Square-free numbers have only $1$s and $0$s.

Multiplication is easy: just add digits. $(7)_{10} \times (18)_{10} = (126)_{10}$ becomes $\ppn{1000} \times \ppn{21} = \ppn{1021}$. $(9)_{10} \times (24)_{10} = (216)_{10}$ becomes $\ppn{20} \times \ppn{13} = \ppn{33}$. Very important: there is no \textit{carry-over} from place to place. To multiply $\ppn{6}$ by $\ppn{15}$ [$64 \times 96 = 6144$ in decimal] will require an additional symbol for the $2$s place: something like $(1\mathrm{B})_p$. In contrast to notations like decimal and binary, which use a fixed repertoire of digits, this notation will need infinitely-many symbols. The Unicode Consortium will not be pleased\ldots.

Addition, however, is a problem. Besides not having an identity element (no zero), the well-known algorithms just don't apply. $\ppn{1} + \ppn{10} = \ppn{100}$ [$2+3=5$] may look promising, but things like $\ppn{21} + \ppn{102} = \ppn{10000001}$ [$18+20=38$] and $\ppn{1000000} + \ppn{10000000} = \ppn{22}$ [$17+19=36$] induce despair. We have also given up being able to sort numbers lexicographically. Numbers with fewer digits are not necessarily smaller than numbers with more. Numbers with the same number of digits can't be reliably sorted without doing calculations. For example, $\ppn{1000} < \ppn{20} < \ppn{12} < \ppn{5}$ is true.

\section*{Divisibility}
If we multiply by adding digit-by-digit, then to divide, we subtract digit-by-digit. The digits in our notation are all non-negative integers, so to divide $b$ by $a$, all of the digits of $a$ must be less than or equal to all of the corresponding digits of $b$, ensuring that all of the differences are non-negative. More formally, $a \mid b$ if and only if every digit of $a$ is less than or equal to the corresponding digit of $b$. (We may, of course, pad our numerals on the left with $0$s as needed.) $\ppn{11} \mid \ppn{13}$ ($6 \mid 24$) because $1 = 1$ and $1 < 3$. Doing the division shows that $\ppn{11} \div \ppn{13} = \ppn{2}$ ($4$ in decimal). $\ppn{20} \nmid \ppn{12}$ ($9 \nmid 12$) because $2 > 1$.

\subsection*{Properties}
The properties of divisibility mainly result from the properties of integers and of multiplication. These properties have explanations in our notation based on the similar properties of addition when applied digit-by-digit. Some examples:

\subsubsection*{$a \mid 0$, $1 \mid a$, $a \mid a$.}
We don't have a zero, so $a \mid 0$ doesn't apply. $1$ is $\ppn{0}$, which divides everything because zero is less than or equal to every non-negative integer. $a \mid a$ because every non-negative integer is less than or equal to itself.

\subsubsection*{$a \mid 1$ if and only if $a = 1$.}
One is $\ppn{0}$. The only way for a number to have all its digits less than or equal to $0$ is if they are all, themselves, $0$. So the number has to be $\ppn{0}$. i.e. $1$.

\subsubsection*{If $a \mid b$ and $c \mid d$, then $ac \mid bd$.}
The digits of $ac$ are the sums of the corresponding digits of $a$ and $c$, as are $bd$ of $b$ and $d$. All the $a$s are less than or equal to all the $b$s, as are the $c$s to the $d$s. So, the $ac$s must be less than or equal to the $bd$s.

\section*{GCDs and LCMs}
\subsubsection*{Greatest Common Divisor}
In order for $d$ to divide both $a$ and $b$, its digits must be less than or equal to the digits of both. The largest number that satisfies this will have each digit equal to the smaller digit of $a$ and $b$. (Any larger, and it wouldn't divide.) For example, to find the greatest common divisor of $(440)_{10}$ and $(528)_{10}$ [$\ppn{10103}$ and $\ppn{10014}$], go digit-by-digit and choose the smaller of each.\\
\ttfamily \small
\addvbuffer[12pt]{\begin{tabular}{r c c c c c}
    & 1 & 0 & 1 & 0 & 3 \\
    & 1 & 0 & 0 & 1 & 4 \\
    \hline
(min) & 1 & 0 & 0 & 0 & 3 \\
\end{tabular}}
\normalfont \normalsize \\
This gives $\ppn{10003}$, which is $(88)_{10}$.

\subsubsection*{Least Common Multiple}
In order for $a$ and $b$ to both divide $m$, both their digits must be less than or equal to the corresponding digit of $m$, so each digit of $m$ must be greater than or equal to the larger digit of $a$ and $b$. The smallest such $m$ will have have each digit equal to the larger of $a$ and $b$. If we look at $(440)_{10}$ and $(528)_{10}$ again:\\
\ttfamily \small
\addvbuffer[12pt]{\begin{tabular}{r c c c c c}
    & 1 & 0 & 1 & 0 & 3 \\
    & 1 & 0 & 0 & 1 & 4 \\
    \hline
(max) & 1 & 0 & 1 & 1 & 4 \\
\end{tabular}}
\normalfont \normalsize \\
we find that the least common multiple is $\ppn{10114}$, or $(2640)_{10}$.

\section*{Primes}

Primes are easy to spot. They all look like $1$ followed by zero or more $0$s. They do get really long (number of digits) really quickly, however. We can get a sense by looking at the function $\pi(n)$, the number of primes not exceeding $n$. $\pi(36) = 11$, so we need $11$ places to show the list of $36$ numbers above. [Prime $(31)_{10}$ is a $1$ followed by ten $0$s.] $\pi(100) = 25$ and $\pi(1000) = 168$. The Prime Number Theorem tells us that we may need as many as (approximately) $n \log n$ digits to display a number $n$ in this notation. The notation doesn't even help us find primes. Sure, you can take a prime and get the next one by adding a $0$ onto the end, but you have to know what prime that new place on the left represents before you can interpret the digits. Since we don't have lexicographic sorting, any two \textit{adjacent} primes look as much like twin primes as not.

A theorem like, “If $p$ is prime and $p \mid ab$, then $p \mid a$ or $p \mid b$.” has a convenient explanation. “$p$ is prime” means that it looks like a $1$ followed by only $0$s. “$p \mid ab$” means that the digit in $ab$ corresponding to the $1$ in $p$ must be at least $1$. That digit is itself the sum of the corresponding digits of $a$ and $b$. For that digit to be at least one, either $a$ or $b$ has that digit be at least one. That is, $p \mid a$ or $p \mid b$.

\section*{Difficulties}

\section*{Code}
Because factoring is hard (\textit{see the previous section}) I needed help converting numbers in and out of this notation. This is the Julia code I used to do that.
\end{document}
